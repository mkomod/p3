\documentclass[12pt]{article}
\usepackage[noend]{algpseudocode}
\usepackage{algorithm}
\usepackage{amsthm}
\usepackage{amsmath}
\usepackage{amsfonts}
\usepackage{amssymb}
\usepackage[round]{natbib}
\usepackage{multirow}
\usepackage{stfloats}
\usepackage{fancyhdr}
\usepackage{float}
\usepackage{graphicx}
\usepackage{xcolor}
\usepackage{geometry}
\usepackage{url}
\usepackage[
  bookmarks=false,
  pdfpagelabels=false,
  hyperfootnotes=false,
  hyperindex=false,
  pageanchor=false,
  colorlinks,
  citecolor=blue
]{hyperref}
\usepackage{cleveref}
\usepackage[notref, notcite]{showkeys}

\graphicspath{{../figures/}}
\linespread{1.3}
\bibliographystyle{abbrvnat}
\newcommand\NoDo{\renewcommand\algorithmicdo{}}     % remove "do" from algs

\setlength{\parindent}{0em}
\setlength{\parskip}{.5em}


\pagestyle{fancy}
\setlength{\headheight}{52pt}





\usepackage{titlesec}
\titlespacing{\section}{0ex}{0.5em}{1em}
\titlespacing{\subsection}{0ex}{0.7em}{0.5em}
\titlespacing{\subsubsection}{0ex}{0.7em}{0.5em}
\titleformat*{\section}{\large \bfseries}
\titleformat*{\subsection}{\normalsize \bfseries}
\titleformat*{\subsubsection}{\normalsize \itshape}


\newcommand{\vs}{

    \vspace{0.5em}

}

\newcommand{\E}{\mathbb{E}}             % Expectation
\newcommand{\R}{\mathbb{R}}             % Real numbers
\newcommand{\I}{\mathbb{I}}             % Indicator function
\newcommand{\D}{\mathcal{D}}            % Calligraphic D
\newcommand{\M}{\mathcal{M}}            % Calligraphic M
\newcommand{\Q}{\mathcal{Q}}            % Calligraphic Q
\renewcommand{\S}{\mathcal{S}}          % Calligraphic S
\renewcommand{\L}{\mathcal{L}}          % Calligraphic L
\renewcommand{\P}{\mathbb{P}}          % Calligraphic L
\newcommand{\KL}{\text{KL}}             % KL divergence
\newcommand{\IG}{\Gamma^{-1}}             % KL divergence
\newcommand{\logistic}{\text{logistic}}             % KL divergence
\newcommand{\sigmoid}{\text{s}}             % KL divergence

\newcommand{\argmin}{{\arg\!\min}}      % arg min without space
\newcommand{\argmax}{{\arg\!\max}}      % arg max without space

\newcommand{\red}[1]{{\color{red} #1}}

\DeclareMathOperator{\card}{card}
\DeclareMathOperator{\corr}{corr}
\DeclareMathOperator{\cov}{cov}
\DeclareMathOperator{\diag}{diag}
\DeclareMathOperator{\rank}{rank}
\DeclareMathOperator{\tr}{tr}
\DeclareMathOperator{\Var}{Var}


\title{Group Sparse Variational Bayes}
\author{}
\date{}

\renewcommand{\red}[1]{\textcolor{red}{#1}}

\begin{document}
\maketitle


\section{Problem formulation}

\subsection{Setting}

Consider the model,
\begin{equation} \label{eq:model} 
    y = \langle x, \beta \rangle + \epsilon
\end{equation}
where $y \in \R$ is the response, $x = (x_1, \dots, x_p)^\top \in \R^p$ a feature vector of explanatory variables, $\beta = (\beta_1, \dots, \beta_p)^\top \in \R^p$ the coefficient vector, and $\epsilon$ a noise term. 

Under a \textit{group-sparse} setting, it is assumed that features can be grouped, and that few groups have non-zero coefficient values \citep{Giraud2021}.  Formally, define the groups as $G_k = \{ G_{k,1}, \dots, G_{k, m_k} \}$ for $k=1,\dots,M$ as disjoint sets of indices such that $ \bigcup_{k=1}^M G_k = \{1, \dots, p \}$ and let $G_k^c = \{1,\dots, p \} \setminus G_k$. Further, denote $x_{G_k} = \{x_j : j \in G_k \}$ and $\beta_{G_k} = \{\beta_j : j \in G_k \}$. 

Under the group structure \eqref{eq:model} can be written as:
\begin{equation}
    y = \left( \sum_{k=1}^{M} \langle x_{G_k}, \beta_{G_k} \rangle \right) + \epsilon
\end{equation}

Furthermore we are going to assume the error term $\epsilon \overset{\text{iid.}}{\sim} N(0, \tau^2)$, under which the log-likelihood is given by,
\begin{align}
    \ell(\D; \beta) = &\ -\frac{1}{2} \sum_{i=1}^n \left[ 
	\log(2\pi \tau^2) + \frac{1}{\tau^2} 
	(y_i - \langle x_i, \beta \rangle)^2
    \right] \nonumber \\
    =&\ - \frac{n}{2}\log(2\pi\tau^2) - \frac{1}{2\tau^2} \| y - X \beta \|^2 \label{eq:log-likelihood}
\end{align}
where $\D = \{ (y_i, x_i) \}_{i=1}^n, y_i \in \R, x_i \in \R^p,  y = (y_1, \dots, y_n)^\top \in \R^n$ and $X = (x_1, \dots, x_n)^\top \in \R^{n \times p}$.


\subsection{Prior}

We consider a group spike-and-slab (GSpSL) prior for the model parameters $\beta$, which has a hierarchical representation,
\begin{equation}
\begin{aligned}
\beta_{G_k} | z_k \overset{\text{ind}}{\sim} &\ z_k \Psi(\beta_{G_k}; \lambda) + (1-z_k) \delta_0(\beta_{G_k}) \\
z_k | \theta_k \overset{\text{ind}}{\sim} &\ \text{Bernoulli}(\theta_k) \\
\theta_k \overset{\text{iid}}{\sim} &\ \text{Beta}(a_0, b_0)
\end{aligned}
\end{equation}
for $k=1,\dots,M$, where $\delta_0$ is the multivariate Dirac mass on zero with dimension $m_k = \dim(\beta_k)$, and $\Psi(\beta_{G_k})$ is the multivariate double exponential distribution with density
\begin{equation} \label{eq:density_mvde}
    \psi(\beta_{G_k}; \lambda) = C_k \lambda^{m_k} \exp \left( - \lambda \| \beta_{G_k} \| \right)
\end{equation}
where $ C_k = \left[ 2^{m_k} \pi^{(m_k -1)/2} \Gamma \left( (m_k + 1) /2 \right) \right]^{-1} $ and $ \| \cdot \| $ is the $\ell_2$-norm. For a visual representation, the density \eqref{eq:density_mvde} for the two-dimensional multivariate double exponential is shown in \Cref{fig:double_exp}.
\begin{figure}[htp]
    \centering
    \includegraphics[width=.9\textwidth]{./figures/de.jpg}
    \caption{Two dimensional double exponential density with $\lambda=1$.}
    \label{fig:double_exp}
\end{figure}

It follows that the prior distribution for $\beta$ is given by,
\begin{equation} \label{eq:prior} 
    \Pi(\beta | z) = \bigotimes_{k=1}^{M} \left[ 
    z_k \Psi(\beta_{G_k}; \lambda) + (1-z_k)\delta_0(\beta_{G_k})
\right]
\end{equation}
where $z = (z_1, \dots, z_M)$ and $\otimes$ is the product measure.


\red{
Regarding $\tau^2$, there are many popular choices that a practitioner may wish to use, for example:
\begin{itemize}
    \itemsep0em
    \item \textit{Locally uniform}, wherein $\tau^2 \sim U(0, 1/\varepsilon)$ for a small positive $\varepsilon$.
    \item \textit{Inverse-Gamma}, wherein $\tau^2 \sim \Gamma^{-1}(a, b)$, where $\Gamma^{-1}$ denotes an inverse-Gamma distribution with shape $a$ and scale $b$, with a common choice for each being $a=b=\varepsilon$, where $\varepsilon$ is a small positive constant around $0.001$.
\end{itemize}
In the meantime, we place do not place a prior on $\tau^2$ and assume it is known.
}

\subsection{Posterior}

The posterior density is given by,
\begin{equation} \label{eq:posterior} 
d\Pi(\beta | \D) = \Pi_D^{-1} e^{\ell(\D; \beta)} d\Pi(\beta)
\end{equation}
where $\Pi_\D = \int_{\R^{p}} e^{\ell(\D; \beta)} d\Pi(\beta)$ is a normalization constant and $\ell(\D; \beta)$ is the log-likelihood function. (This isn't exactly correct, we still need to integrate our the $z$ terms and re-write the prior).


\subsection{Variational Family}

In turn, our aim is to approximate \eqref{eq:posterior} by a member of a tractable family of distributions, referred to as the variational family. We have chosen the family,
\begin{equation}
    \Q = \left\{ Q =  \bigotimes_{k=1}^M \left[ 
    \gamma_k\ N\left(\mu_{G_k}, \diag( \sigma_{G_k}) \right) + (1-\gamma_k) \delta_0
\right] \right\}
\end{equation}
where $N(\mu, \Sigma)$ denotes the multivariate Normal distribution with mean parameter $\mu$ and covariance $\Sigma$. In turn the variational posterior is given by solving,
\begin{equation} \label{eq:optim} 
\tilde{\Pi} = \underset{Q \in \Q}{\argmin}\ \KL\left( Q \| \Pi(\cdot |\D) \right)
\end{equation}
and is used in subsequent analysis as a proxy for the true posterior.


% ----------------------------------------
% ----------------------------------------
% ----------------------------------------
\section{Co-ordinate ascent algorithm}

Throughout our derivation we exploit the group independence structure within the prior and variational distribution, allowing the Radon-Nikodym derivative of $Q$ with respect to the prior $\Pi$ to be expressed as,
\begin{equation}
    \frac{dQ}{d\Pi}(\beta) = \prod_{k=1}^{M} \frac{dQ_k}{d\Pi_k} (\beta_{G_k})
\end{equation}

Consider, the optimization problem \eqref{eq:optim} and recall the definition of the KL divergence \eqref{eq:kl}, it follows that
\begin{align}
    \KL(Q \| \Pi(\cdot | \D)) 
    = &\ \E_{Q} \left[ \log \frac{dQ}{d\Pi(\dot | \D)} \right] 
    = \E_Q \left[ \log \frac{\Pi_\D dQ}{e^{\ell(\D; \beta)} d\Pi} \right] \nonumber \\
    =&\ \E_Q \left[ -\ell(\D; \beta) + \log \frac{dQ}{d\Pi} \right] + \log \Pi_\D \label{eq:opt} 
\end{align}
As optimization of the objective \eqref{eq:optim} is invariant to constant terms, to simplify upcoming expressions we write them as $C$ (the value of which may change line by line).

\subsection{Updates of $\mu_{G_k}$ and $\sigma_{G_k}$}

In order to update $\mu_{G_k}$ and $\sigma_{G_k}$ we must assume that the group takes a non-zero value, i.e. $z_k =1$. Hence, %
{\allowdisplaybreaks
\begin{align}
\E_{Q  | z_K = 1} & \left[ 
    - \ell(\D; \beta) + \log \frac{dQ}{d\Pi}(\beta) 
\right]  \nonumber \\
= &\
    \E_{Q | z_K = 1} \left[ 
	- \ell(\D; \beta) + \log \prod_{k=1}^M \frac{dQ_k}{d\Pi_k}(\beta_{G_k})
    \right] \nonumber \\
= &\
    \E_{Q | z_K = 1} \left[ 
	- \ell(\D; \beta) 
	+ \log \frac{dQ_K}{d\Pi_K}(\beta_{G_K})
	+ \log \prod_{k \neq K} \frac{dQ_k}{d\Pi_k}(\beta_{G_k})
    \right] \nonumber \\
= &\
    \E_{Q | z_K = 1} \left[ 
	\frac{1}{2\tau^2} \| y - X \beta \|^2
	+ \log \frac{dQ_K}{d\Pi_K}(\beta_{G_K})
    \right] + C \nonumber \\
= &\
    \E_{Q | z_K = 1} \left[ 
	\frac{1}{2\tau^2} \bigg\{ 
	    \| X \beta \|^2 - 2 \langle y, X\beta \rangle 
	\bigg\}
	+ \log \frac{dQ_K}{d\Pi_K}(\beta_{G_K})
    \right] + C \nonumber \\
= &\
    \E_{Q | z_K = 1} \left[ 
	\frac{1}{2\tau^2} \left\{ 
	    \tr \left( X^\top X \beta \beta^\top \right) 
	    - 2 \sum_{k=1}^M \langle y, X_{G_k} \beta_{G_k} \rangle 
	\right\}
	+ \log \frac{dQ_K}{d\Pi_K}(\beta_{G_K})
    \right] + C \nonumber \\
= &\
    \E_{Q | z_K = 1} \left[ 
	\frac{1}{2\tau^2} \tr \left( X^\top X \beta \beta^\top \right) 
	- \frac{1}{\tau^2} \langle y, X_{G_K} \beta_{G_K} \rangle 
	+ \log \frac{dQ_K}{d\Pi_K}(\beta_{G_K})
    \right] + C \label{eq:mu_sigma_1}
\end{align}
} %
where $\tr(\cdot)$ denotes the trace of a matrix and $C$ is a constant term whose value does not depend on $\mu_{G_K}$ or $\sigma_{G_K}$ (and may change line by line).

Consider the matrix $ X^\top X \beta \beta^\top \in \R^{p \times p} $, using the fact that $\left( X^\top X \beta \beta^\top \right)_{ii} = \sum_{j=1}^{p} (X^\top X)_{ji} \beta_j \beta_i $ for $i,j = 1, \dots, p$, we have
\begin{align}
    \E_{Q | z_K = 1} & \left[ 
	\tr \left( X^\top X \beta \beta^\top \right) 
    \right]
=
    \E_{Q | z_K = 1} \left[ 
	\sum_{i=1}^p \sum_{j=1}^{p} (X^\top X)_{ji} \beta_j \beta_i 
    \right] \nonumber \\
=&\
    \sum_{i=1}^p \sum_{j=1}^{p} (X^\top X)_{ji} 
    \E_{Q | z_K = 1} \left[ \beta_j \beta_i \right] 
    \nonumber \\
=&\
    \sum_{i \in G_K} \left(
	\sum_{j=1}^{p} (X^\top X)_{ji} 
	\E_{Q | z_K = 1} \left[ \beta_j \beta_i \right]
    \right)
+
    \sum_{i \in G_K^c} \left(
	\sum_{j=1}^{p} (X^\top X)_{ji} 
	\E_{Q | z_K = 1} \left[ \beta_j \beta_i \right] 
    \right)
    \nonumber \\
=&\
    \sum_{i \in G_K} \left( 
	\sum_{j \in G_K} (X^\top X)_{ji} 
	    \E_{Q | z_K = 1} \left[ \beta_j \beta_i \right] 
	+ 
	\sum_{j \in G_K^c} (X^\top X)_{ji} 
	    \E_{Q | z_K = 1} \left[ \beta_j \beta_i \right] 
    \right ) \nonumber \\
+&\ 
    \sum_{i \in G_K^c} \left( 
	\sum_{j \in G_K} (X^\top X)_{ji} 
	    \E_{Q | z_K = 1} \left[ \beta_j \beta_i \right] 
	+ 
	\sum_{j \in G_K^c} (X^\top X)_{ji} 
	    \E_{Q | z_K = 1} \left[ \beta_j \beta_i \right] 
    \right ) \nonumber \\
=&\
    \sum_{i \in G_K} \left( 
	\sum_{j \in G_K} (X^\top X)_{ji} 
	    \E_{Q | z_K = 1} \left[ \beta_j \beta_i \right] 
	+ 
	2 \sum_{j \in G_K^c} (X^\top X)_{ji} 
	    \E_{Q | z_K = 1} \left[ \beta_j \beta_i \right] 
    \right ) + C \nonumber
\end{align}
Consider, $ \E_{Q | z_K = 1} \left[ \beta_j \beta_i \right] $ and note $\cov(\beta_j, \beta_i) = \E\left[ \beta_j \beta_i \right] - \E \left[\beta_j \right] \E \left[ \beta_i \right]$, the previous display results in three cases
\begin{equation}
    \E_{Q | z_K = 1} \left[ \beta_j \beta_i \right] = \begin{cases}
	\sigma_{j}^2 + \mu_{j}^2 	& \quad i,j \in G_K, i=j \\
	\mu_{j}\mu_{i} 			& \quad i,j \in G_K, i \neq j \\
	\gamma_J \mu_{j}\mu_{i} 	& \quad i \in G_K, j \in G_J, J \neq K
    \end{cases}
\end{equation}

The second term in \eqref{eq:mu_sigma_1} is straightforward and is given as,
\begin{equation} \label{eq:mu_sigma_term_2}
    \E_{Q | z_K = 1} \left[ \langle y, X_{G_K} \beta_{G_K} \rangle  \right]
=   
    \langle y, X_{G_K} \E_{Q | z_K = 1} \left[ \beta_{G_K} \right] \rangle   
=
    \langle y, X_{G_K} \mu_{G_K} \rangle  
\end{equation}

Finally, the third term in \eqref{eq:mu_sigma_1} is given as,
\begin{align}
    \E_{Q | z_K = 1 } & \left[ \log \frac{dQ_K}{d\Pi_K} (\beta_{G_K}) \right]
    \nonumber \\
= &\
    \E_{Q | z_K = 1 } \left[ 
	\log \frac
	{\prod_{j \in G_K} \left( 2 \pi \sigma_j^2 \right)^{-1/2} \exp \left\{ -(2\sigma^2_{j})^{-1}(\beta_{j} - \mu_j )^2 \right\}}
	{C_K \lambda^{m_K} \exp\left( - \lambda \| \beta_{G_K} \| \right)}
    \right] \nonumber \\
= &\
    \E_{Q | z_K = 1 } \left[ 
	\lambda \| \beta_{G_K} \|
	- \sum_{j \in G_K} \left( 
	    \log{\sigma_j} 
	    + \frac{1}{2\sigma_j^2} (\beta_{j} - \mu_j)^2
	\right)
    \right] + C \nonumber \\
= &\
    \lambda \E_{Q | z_K = 1 } \left[ 
	\| \beta_{G_K} \|
    \right] 
    - \sum_{j \in G_K} \log{\sigma_j} 
    - \frac{m_K}{2} 
    + C 
    \label{eq:mu_simga_term_3}
\end{align}
Evaluating the remaining expectation in \eqref{eq:mu_simga_term_3} is non-trivial, in turn we derive an upper bound using Jensen's inequality,
\begin{equation} \label{eq:mu_sigma_upper}
    \E_{Q | z_K = 1 } \left[ 
	\| \beta_{G_K} \|
    \right] 
    % \E_{Q | z_K = 1 } \left[ 
	% \left( \sum_{j \in G_K} \beta_{j}^2 \right)^{1/2}
    % \right] 
    % \nonumber \\
\leq
    \left( \sum_{j \in G_K} 
	\E_{Q | z_K = 1 } \left[ \beta_{j}^2 \right] 
    \right)^{1/2} 
=
    \left( \sum_{j \in G_K} 
	\sigma_j^2 + \mu_j^2
    \right)^{1/2} 
\end{equation}
Putting these components together gives
\begin{equation} \label{eq:mu_sigma_main}
\begin{aligned}
    \E_{Q  | z_K = 1} & \left[ 
	- \ell(\D; \beta) + \log \frac{dQ}{d\Pi}(\beta) 
    \right]  \\
\leq &\
    \frac{1}{2\tau^2} 
    \sum_{i \in G_K} \left( 
	    (X^\top X)_{ii} (\sigma_i^2 + \mu_i^2)
	+
	\sum_{j \in G_K, j\neq i} 
	    (X^\top X)_{ji} \mu_j \mu_i
    \right ) \\
+ &\
    \frac{1}{\tau^2} 
    \sum_{i \in G_K} \left( 
    \sum_{j \in G_K^c} (X^\top X)_{ji} 
	\gamma_{J} \mu_j \mu_i
    \right )
-
    \frac{1}{\tau^2}
    \langle y, X_{G_K} \mu_{G_K} \rangle   \\
- &\
    \sum_{i \in G_K} \log{\sigma_i}
+
    \lambda \left( \sum_{i \in G_K} 
	\sigma_i^2 + \mu_i^2
    \right)^{1/2} + C
\end{aligned}
\end{equation}

\subsubsection{Group-wise update for $\mu_{G_K}$}

Re-writing the RHS of \eqref{eq:mu_sigma_main} in terms of $\mu_{G_K}$, we have
\begin{equation} \label{eq:mu_gk}
\begin{aligned}
    \frac{1}{2\tau^2} 
    \mu_{G_K}^\top X_{G_K}^\top X_{G_K} \mu_{G_K}
+
    \frac{1}{\tau^2} 
    \sum_{J \neq K} 
	\gamma_J \mu_{G_K}^\top X_{G_K}^\top X_{G_J} \mu_{G_J} 
-
    \frac{1}{\tau^2}
    \langle y, X_{G_K} \mu_{G_K} \rangle \\
+
    \lambda \left( \sigma_{G_K}^\top \sigma_{G_K} + \mu_{G_K}^\top \mu_{G_K} \right)^{1/2} + C
\end{aligned}
\end{equation}

\textbf{Insight into the update of $\mu_{G_K}$}

Using the fact that $ \left( \sum_{j \in G_K} \sigma_j^2 + \mu_j^2 \right)^{1/2} \leq 1 + \sum_{j \in G_K} \sigma_j^2 + \mu_j^2 $, which follows from $x \leq 1 + x^2$, we can upper bound \eqref{eq:mu_gk}.
\begin{equation} \label{eq:mu_gk_upper}
\begin{aligned}
    \frac{1}{2\tau^2} 
    \| X_{G_K} \mu_{G_K} \|^2
+
    \frac{1}{\tau^2} 
    \sum_{J \neq K} 
	\gamma_J \mu_{G_K}^\top X_{G_K}^\top X_{G_J} \mu_{G_J} 
-
    \frac{1}{\tau^2}
    \langle y, X_{G_K} \mu_{G_K} \rangle
+
    \lambda \| \mu_{G_K} \|^2 + C
\end{aligned}
\end{equation}
And in turn \eqref{eq:mu_gk_upper} is minimized by
\begin{equation} \label{eq:mu_gk_min}
    \hat{\mu}_{G_K} = \Xi^{-1} X_{G_K}^\top y - \Xi^{-1} \sum_{J \neq K} \gamma_J X_{G_K}^\top X_{G_J} \mu_{G_J}    
\end{equation}
where $\Xi = X_{G_K}^\top X_{G_K} + 2 \lambda \tau^2 I_{m_K}$. 

Let $P := (X_{G_K}^\top X_{G_K} + 2 \lambda \tau^2 I_{m_K})^{-1} X_{G_K}^\top $ and the prediction of $y$ from $\mu_{G_J}$ as $\hat{y}_{G_J} = X_{G_J} \mu_{G_J}$, then we can re-express \eqref{eq:mu_gk_min} as
\begin{equation}
\hat{\mu}_{G_K} = P(y - \sum_{J \neq K} \gamma_J \hat{y}_{G_J})
\end{equation}
In other words the minimizer $\hat{\mu}_{G_K}$ seeks a vector that explains the remaining signal in $y$ given the signal explained by $\sum_{J \neq K} \gamma_J \hat{y}_{G_J}$, for example, consider the extreme case, $y - \sum_{J \neq K} \gamma_J \hat{y}_{G_J} = 0_n$ (the n-dimensional zero vector), then the resulting minimizer $\hat{\mu}_{G_K} = 0_{m_K}$. 

It turns out, updating using \eqref{eq:mu_gk_min} doesn't work that well in practise, in fact, using optimization routines to minimize \eqref{eq:mu_gk} leads to better results in less time.


\subsubsection{Group-wise update for $\sigma_{G_K}$}

Re-writing the RHS of \eqref{eq:mu_sigma_main} in terms of $\sigma_{G_K}$, we have
\begin{equation} \label{eq:sig_gk}
\begin{aligned}
    \sum_{i \in G_K} \left( 
    \frac{1}{2\tau^2} 
	    (X^\top X)_{ii} \sigma_i^2
-
    \log{\sigma_i}
    \right )
+
    \lambda \left( \sum_{i \in G_K} 
	\sigma_i^2 + \mu_i^2
    \right)^{1/2} + C
\end{aligned}
\end{equation}

\subsection{Updates for $\gamma_K$}

Similarly for $\gamma_K$ we evaluate the expectation with respect to $Q$, however without conditioning on the group being non-zero.
{\allowdisplaybreaks
\begin{align}
    \E_{Q} & \left[ 
	- \ell(\D; \beta) + \log \frac{dQ}{d\Pi}(\beta) 
    \right]  \nonumber \\
= &\
    \E_{Q} \left[ 
	- \ell(\D; \beta) 
	+ \I_{\{z_K = 1\}} \log \frac{\gamma_K dN_K}{\bar{w} d\Psi_K}(\beta_{G_K}) 
	+ \I_{\{z_K = 0\}} \log \frac{1 - \gamma_K}{1 - \bar{w}}
    \right] + C \nonumber \\
=&\
    \frac{1}{2\tau^2} 
    \sum_{i \in G_K} \left( 
	\sum_{j \in G_K} (X^\top X)_{ji} 
	    \E_{Q} \left[ \beta_j \beta_i \right] 
	+ 
	2 \sum_{j \in G_K^c} (X^\top X)_{ji} 
	    \E_{Q} \left[ \beta_j \beta_i \right] 
    \right ) \nonumber \\
- &\
    \frac{1}{\tau^2} \langle y, X_{G_K} \E_Q\left[\beta_{G_K} \right] \rangle 
-
    \frac{\gamma_K}{2} \sum_{j \in G_K} \log \left( 2 \pi \sigma_j^2 \right)
-
    \gamma_K \log(C_K )
    \nonumber \\
- &\
    \gamma_K m_K \log (\lambda) 
+ 
    \E_{Q} \left[ 
	\I_{\{z_K=1\}} \left(
	\lambda \| \beta_{G_K} \|
	- \sum_{j \in G_K}
	    \frac{1}{2\sigma_j^2} (\beta_{j} - \mu_j)^2
	\right)
    \right]  \nonumber \\ 
+ &\
    \gamma_K \log \frac{\gamma_K}{\bar{w}}
    + (1 - \gamma_K) \log \frac{1 - \gamma_K}{1 - \bar{w}}
+ C \nonumber
\end{align}
}
Noting $\E_Q \left[ \beta_{G_K} \right] = \gamma_K \mu_{G_K} $ and
\begin{equation}
    \E_{Q} \left[ \beta_j \beta_i \right] = \begin{cases}
	\gamma_K (\sigma_{j}^2 + \mu_{j}^2) 	& \quad i,j \in G_K, i=j \\
	\gamma_K \mu_{j}\mu_{i} 		& \quad i,j \in G_K, i \neq j \\
	\gamma_K \gamma_J \mu_{j}\mu_{i} 	& \quad i \in G_K, j \in G_J, J \neq K
    \end{cases}
\end{equation}
and
\begin{equation}
    \E_Q \left[ \I_{\{z_K = 1\}} \| \beta_{G_K} \| \right] = 
    \gamma_K \E_{N_K} \left[ \| \beta_{G_K} \| \right]
    \leq \gamma_K \left( \sum_{j \in G_K} \sigma^2_j + \mu^2_j \right)^{1/2}
\end{equation}
Substituting these expressions into the previous display gives,
\begin{equation*}
\begin{aligned}
    \E_{Q} & \left[ 
	- \ell(\D; \beta) + \log \frac{dQ}{d\Pi}(\beta) 
    \right]  \\
\leq &\
    \frac{\gamma_K}{2\tau^2}
    \sum_{i \in G_K} \left( 
	    (X^\top X)_{ii} (\sigma_i^2 + \mu_i^2)
	+
	\sum_{j \in G_K, j\neq i} 
	    (X^\top X)_{ji} \mu_j \mu_i
    \right ) \\
+ &\
    \frac{\gamma_K}{\tau^2}
    \sum_{i \in G_K} \left( 
    \sum_{j \in G_K^c} (X^\top X)_{ji} 
	\gamma_{J} \mu_j \mu_i
    \right )
-
    \frac{\gamma_K}{\tau^2} \langle y, X_{G_K} \mu_{G_K} \rangle \\
- &\
    \frac{\gamma_K}{2} \sum_{j \in G_K} \log \left( 2 \pi \sigma_j^2 \right) 
-
    \gamma_K \log(C_K )
-
    \gamma_K m_K \log (\lambda) 
+
    \lambda \gamma_K \left( \sum_{j \in G_K} 
	\sigma_j^2 + \mu_j^2
    \right)^{1/2} \\
- &\
    \frac{\gamma_K m_K}{2} 
% + 
    % \gamma_K \sum_{j \in G_K} \frac{\mu_j^2}{2 \sigma_j^2}
+ 
    \gamma_K \log \frac{\gamma_K}{\bar{w}}
+ 
    (1 - \gamma_K) \log \frac{1 - \gamma_K}{1 - \bar{w}}
+ C
\end{aligned}
\end{equation*}
Differentiating the RHS of the previous display with respect to $\gamma_K$, setting to zero and re-arranging gives the update equation for $\gamma_K$, formally,
\begin{equation} \label{eq:update_gamma} 
\begin{aligned}
    \log &\ \frac{\gamma_K}{1-\gamma_K} = 
    \log \frac{\bar{w}}{1-\bar{w}}
+ 
    \frac{m_K}{2}  
+
    \frac{1}{\tau^2} \langle y, X_{G_K} \mu_{G_K} \rangle  \\
+ &\ 
    \frac{1}{2} \sum_{j \in G_K} \log \left( 2 \pi \sigma_j^2 \right)
+
    \log(C_K )
+
    m_K \log (\lambda)
-
\Bigg\{ 
    \lambda \left( \sum_{j \in G_K} 
	\sigma_j^2 + \mu_j^2
    \right)^{1/2}  \\
+ &\
    \frac{1}{2\tau^2}
    \sum_{i \in G_K} \left( 
    (X^\top X)_{ii} \sigma_i^2
    +
    \sum_{j \in G_K} 
	(X^\top X)_{ji} \mu_j \mu_i
    % \right )
+
    % \frac{1}{\sigma^2}
    % \sum_{i \in G_K} \left( 
    2 \sum_{j \in G_K^c} (X^\top X)_{ji} 
	\gamma_{J} \mu_j \mu_i
    \right )
\Bigg\}
\end{aligned}
\end{equation}


\subsection{Evidence Lower Bound}

The evidence lower bound, acts as a lower bound for the model evidence $\Pi_\D$, and follows from the definition of the KL divergence,
\begin{align*}
 & 0 \leq \KL(Q \| \Pi(\cdot | \D)) = \E_Q \left[ \Pi_\D - \ell(\D; \beta) - \log \frac{d\Pi}{dQ} (\beta) \right] \\
    \implies & \E_Q \left[ \ell(\D; \beta) + \log \frac{d\Pi}{dQ} (\beta) \right] \leq \Pi_\D
\end{align*}
Formally, the ELBO is defined as,
\begin{equation}
   \L(\D) = \E_Q \left[ \ell(\D; \beta) + \log \frac{d\Pi}{dQ}(\beta) \right]
\end{equation}
In effect, maximizing the ELBO is equivalent to minimizing $\KL(Q \| \Pi(\cdot | \D)) $, i.e. solving \eqref{eq:optim}. Often the ELBO is used to assess the convergence of co-ordinate ascent algorithms, but can also act as a goodness of fit measure.

For our model the ELBO is given as
\begin{equation} \label{eq:elbo} 
\begin{aligned}
    \L(\D) &= 
- 
    \frac{n}{2} \log(2 \pi \tau^2) 
- 
    \frac{1}{2\tau^2} \left( \| y \|^2  + \sum_{i=1}^p \sum_{j=1}^p (X^\top X)_{ji} \E_Q \left[ \beta_j \beta_i \right] \right)\\
+& 
    \sum_{K=1}^M \bigg(  
\frac{1}{\tau^2} \gamma_K \langle y, X_{G_K} \mu_{G_K} \rangle
+
    \frac{\gamma_K}{2} \sum_{j \in G_K} \left( \log( 2 \pi \sigma_j^2) \right)
+
    \gamma_K \log(C_K)
+ 
    \frac{\gamma_K m_K}{2} \\
+ &
    \gamma_K m_K \log (\lambda)
-
    \E_Q \left[ \I_{z_K =1} \lambda \| \beta_{G_K} \| \right]
-
    \gamma_K \log \frac{\gamma_K}{\bar{w}}
-
    (1-\gamma_K) \log \frac{1 - \gamma_K}{1 - \bar{w}}
\bigg)
\end{aligned}
\end{equation}
where the expectation $\E_Q \left[ \beta_i \beta_j \right]$ is defined earlier. Given $\E_Q \left[ \I_{\{z_K = 1\}} \lambda \| \beta_{G_K} \| \right] $ does not have a closed form expression, Monte-Carlo integration to evaluate it.

\subsection{Updating $\tau$}

As of yet we have not placed a prior on $\tau^2$ or considered updating $\tau^2$ as part of our co-ordinate ascent algorithm. This is because, under any independent prior for $\tau^2$ and independent factorization in the mean-field variational family, we can modify (without major alterations) our existing equations for $\mu_{G_K}, \sigma_{G_K}$, $\gamma_K$ and ELBO to include this modelling assumption.

We opt to use an inverse Gamma prior, predominately because it is a popular choice amongst practitioners \citep{Browne2006}. Formally, we let,
\begin{equation}
    \tau^2 \sim \Gamma^{-1}(a, b)
\end{equation}
where $\Gamma^{-1}(a, b)$ denotes the inverse Gamma distribution with density $\frac{b^a}{\Gamma(a)}x^{-a-1} e^{-b/x}$. Extending our prior \eqref{eq:prior} to include this modelling assumption for $\tau^2$, we have,
\begin{equation}
    \Pi'(\beta, \tau^2) = \Pi(\tau^2) \Pi(\beta) = \Gamma^{-1}(\tau^2; a, b) \Pi(\beta)
\end{equation}
Similarly, extending our variational family with respect to $\tau^2$, we have,
\begin{equation}
    \Q' = \{ \Gamma^{-1}(\tau^2; a', b') : a' > 0, b' > 0 \} \times \Q
\end{equation}
where we denote,
\begin{equation}
    Q' = \Gamma^{-1}(\tau^2; a', b') \otimes Q(\beta) \in \Q' 
\end{equation}

Given these extension, our original optimization problem \eqref{eq:opt} is extended to be
\begin{equation}
    \E_{Q'} \left[ -\ell(\D; \beta) + \log \frac{dQ'}{d\Pi'} \right] = 
    \E_{Q'} \left[ -\ell(\D; \beta) + \log \frac{dQ}{d\Pi} + \log \frac{d\Gamma^{-1}(a', b')}{d \Gamma^{-1}(a, b)}(\tau^2)\right]
\end{equation}
Consequently, when deriving the new update equations for $\mu_{G_K}, \sigma_{G_K}$ and $\gamma_K$, this new term $ \E [ \log \frac{d\Gamma^{-1}}{d\Gamma^{-1}} ] $ is a constant and can be ignored. Therefore, we need only replace all occurrences of $1/\tau^2$ in our old update equations with $\E_{\Gamma^{-1}} [ 1 / \tau^2 ] = a'/b' $. Regarding the ELBO we do the same, however we also include the term $ \E_{\Gamma^{-1}(a', b')} [ \log \frac{d\Gamma^{-1}(a', b')}{d\Gamma^{-1}(a, b)}]$ which is given by,
\begin{equation}
    a' \log (b') - a \log(b) + \log \frac{\Gamma(a)}{\Gamma(a')} + (a-a')(\log(b') + \kappa(a')) + (b - b')\frac{a'}{b'}
\end{equation}
noting $\E_{\Gamma^{-1}} [ \log(\tau^2) ] =  \log(b') + \kappa(a') $  where $\kappa(\cdot)$ is the digamma function. 

To update $a'$ and $b'$ we follow a similar procedure as before, writing \eqref{eq:opt} as a function of $a'$ and $b'$ and then finding the minimum,
\begin{align}
    \E_{Q'} & \left[ -\ell(\D; \beta) + \log \frac{dQ'}{d\Pi'} \right]
=  
    \E_{Q'} \left[
-
    \ell(\D; \beta)
+   
    \log \frac{d\Gamma^{-1}(a', b')}{d \Gamma^{-1}(a, b)}(\tau^2)\right] 
+
    C \nonumber \\
= &\
\E_{Q'} \left[ 
    \frac{n}{2}\log(\tau^2) 
+ 
    \frac{1}{2\tau^2} \| y - X \beta \|^2 
\right]
+
\E_{\Gamma^{-1}} \left[
    \log \frac{d\Gamma^{-1}(a', b')}{d \Gamma^{-1}(a, b)}(\tau^2)
\right] + C \nonumber \\
= &\
\E_{\Gamma^{-1}} \left[
    \frac{1}{2\tau^2}
\right]
\E_{Q} \left[ 
 \| y - X \beta \|^2 
\right]
+
\E_{\Gamma^{-1}} \left[
    \frac{n}{2}\log(\tau^2) 
+ 
    \log \frac{d\Gamma^{-1}(a', b')}{d \Gamma^{-1}(a, b)}(\tau^2)
\right] + C \nonumber
\end{align}
\begin{equation} \label{eq:update_a_b} 
\begin{aligned}
= & \
\frac{a'}{2 b'} \left( \| y \|^2 
    - 2 \langle y, X \E_Q \left[ \beta \right] \rangle
+  \sum_{i=1}^p \sum_{j=1}^p (X^\top X)_{ji} \E_Q \left[ \beta_j \beta_i \right]
\right)
+ a' \log (b')  \\
- &\
\log \Gamma(a') + \left( \frac{n}{2} + a-a' \right)(\log(b') + \kappa(a')) + (b - b')\frac{a'}{b'} + C
\end{aligned}
\end{equation}
\red{Note: finding the values of $a'$ and $b'$ that minimizes \eqref{eq:update_a_b} needs to be done jointly. This is contrary to the outline of CAVI where parameters are optimized independently of one another. We've seen this before when optimizing $\mu_{G_k}$. For instance, optimizing each element of $\mu_{G_k}$ independently of the rest does not lead to a stable algorithm.
}

\red{
Q: why is it the case that some parameters need to be optimized jointly and others do not? And when do they need to be optimized together or can be optimized separately. \\ 
H: maybe this has something to do with how the first moment is defined, i.e. the first moment of $\tau^2$ is given by $b' / (a' - 1)$ which depends on both $a'$ and $b'$. \\
Q: does this hold for certain types of distributions? i.e. do exponential family distributions with multiple parameter means need to have those parameters optimized jointly?
}

% Writing the previous display as a function of $a'$ we have
% \begin{equation} \label{eq:a_update}
% \begin{aligned}
% & \frac{a'}{2b'} 
%     \left( \| y \|^2 
% - 
%     2 \sum_{K=1}^M \gamma_{G_K} \langle y, X_{G_K} \mu_{G_K} \rangle
% +  
%     \sum_{i=1}^p \sum_{j=1}^p (X^\top X)_{ji} \E_Q \left[\beta_j \beta_i \right]
%     \right) 
% \\
% - &\ \log \Gamma(a') + \left( \frac{n}{2} + a - a' \right)\kappa(a') + (b - b')\frac{a'}{b'} + C
% \end{aligned}
% \end{equation}
% Similarly, for $b'$ we have,
% \begin{equation} \label{eq:b_opt}
% \begin{aligned}
% & \frac{a'}{2b'} 
%     \left( \| y \|^2 
% - 
%     2 \sum_{K=1}^M \gamma_{G_K} \langle y, X_{G_K} \mu_{G_K} \rangle
% +  
%     \sum_{i=1}^p \sum_{j=1}^p (X^\top X)_{ji} \E_Q \left[\beta_j \beta_i \right]
%     \right) 
% \\
% + &\ \left( \frac{n}{2} + a \right)\log(b') + \frac{ba'}{b'} + C
% \end{aligned}
% \end{equation}
% As there is no closed form minimizer of \eqref{eq:a_update} optimization routines must be used to find the minimum, whereas the minimizer of \eqref{eq:b_opt} is given by,
% \begin{equation*}
%     b'_{\min} = 
% \frac{a'}{2}
% \frac{n/2 +a - a'}
% {
%     \| y \|^2 
% - 
%     2 \sum_{K=1}^M \gamma_{G_K} \langle y, X_{G_K} \mu_{G_K} \rangle
% -
%     2b
% +  
%     \sum_{i=1}^p \sum_{j=1}^p (X^\top X)_{ji} \E_Q \left[\beta_j \beta_i \right]
% }
% \end{equation*}


\subsection{Implementation details}

Until convergence repeat the following steps:
\begin{enumerate}
    \item For $k=1,\dots,M$
    \begin{enumerate}
	\item Update $\mu_{G_k} \leftarrow \underset{\mu_{G_k} \in \R^{m_k}}{\argmin}\ f(\mu_{G_k}; \mu_{G_k^c}, \sigma, \gamma, \tau)$
	\item Update $\sigma_{G_k} \leftarrow \underset{\sigma_{G_k} \in \R^{m_k}}{\argmin}\ g(\mu_{G_k}; \mu_{G_k^c}, \sigma, \gamma, \tau)$
	\item Update $\gamma_k \leftarrow \text{sigmoid} \ h(\mu_{G_k}; \mu_{G_k^c}, \sigma, \gamma, \tau)$
	\item Update $a', b'$ minimizers of \eqref{eq:update_a_b}
    \end{enumerate}
\end{enumerate}

The algorithm can be sensitive to initialization, in our implementation we used the group LASSO from the package \texttt{gglasso} to initialize $\mu$. To initialize $\sigma_k$ 






\newpage
\section{Simulation study}

\subsection{Simulation design}

Data is simulated for $i=1,\dots,n$ observations, each having a response $y_i \in \R$ and $p$ continuous predictors $x_i \in \R^p$. The response is sampled independently from a Gaussian distribution with mean $\beta_0^\top x_i$ and variance $\sigma^2$, where the true coefficient vector $\beta_0 = (\beta_{0, 1}, \dots, \beta_{0, p})^\top \in \R^p$ contains $s$ non-zero groups each of size $g$. Non-zero elements of $\beta_{0}$ are sampled independently and uniformly from $[-3,-1] \cup [1,3]$. Finally, the predictors are generated from one of \red{\#} settings:
\begin{itemize}
    \item \textbf{Setting 1}: $x_i \overset{\text{iid}}{\sim} N(0_p, I_p)$ where $0_p$ is the p-dimension zero vector and $I_p$ the $p\times p$ identity matrix.
    \item \textbf{Setting 2}: $x_i \overset{\text{iid}}{\sim} N(0, \Sigma)$ where $\Sigma_{ij} = 0.6^{|i - j|}$ for $i,j=1,\dots,p$.
    \item \textbf{Setting 3}: $x_i \overset{\text{iid}}{\sim} N(0, \Sigma)$  where $\Sigma_{ii} = 1$, $\Sigma_{ij}=0.6$ for $i\neq j$ and $i, j = 50k, \dots, 50(k+1)$ for $k=0,\dots, p/50 -1$ and $\Sigma_{ij} =0$ otherwise.
\end{itemize}

\subsection{Methods}

Considered so far:
\begin{itemize}
    \item \textbf{GSVB} (group sparse variational Bayes -- ours): this is an implementation of
    \item \textbf{MCMC}: Gibb sampler for the group spike-and-slab prior
    \item \textbf{SSGL}: Spike-and-slab group LASSO, (similar to the spike-and-slab LASSO) where the multivariate Dirac mass is replaced with a multivariate double exponential distribution. (i.e. this is a continuous mixture), with one density acting as the spike and another the slab.
\end{itemize}


\subsection{Results}

Methods are comparable. Our method has the fastest runtime, however this may come down to the fact that our method (and the MCMC implementation) is written in C\texttt{++}, whereas GSVB is written in \texttt{R}.

\begin{table}[htp]
    \centering
    \resizebox{\textwidth}{!}{ %
% {\setlength{\tabcolsep}{2.0em} 
{\setlength{\tabcolsep}{1.0em} 
\begin{tabular}{| l l | c c c c c c |}
    \hline
    Setting & Method & $\ell_2$-error & $\ell_1$-error & TPR & FDR & AUC & Runtime \\
    \hline
    % --- Indep
    \rule{0pt}{1\normalbaselineskip}
    \multirow{3}{*}{\textit{Setting 1}}
    & GSVB &
 0.269 (0.18, 0.36)& 0.851 (0.59, 1.14)& 1.000 (1.00, 1.00)& 0.000 (0.00, 0.00)& 1.000 (1.00, 1.00)&1.4s (1.3s, 2.1s)\\
    & MCMC &
 0.272 (0.18, 0.36)& 0.845 (0.58, 1.13)& 1.000 (1.00, 1.00)& 0.000 (0.00, 0.00)& 1.000 (1.00, 1.00)&3m 14s (3m 10s, 5m 39s)\\
    & GSSL &
 0.273 (0.19, 0.36)& 0.844 (0.59, 1.15)& 1.000 (1.00, 1.00)& 0.000 (0.00, 0.00)& 1.000 (1.00, 1.00)&11.8s (10.7s, 13.7s)\\
    [.4em]
    \hline
    % --- Diag
    \rule{0pt}{1\normalbaselineskip}
    \multirow{3}{*}{\textit{Setting 2}}
    & GSVB &
 0.367 (0.24, 0.51)& 1.143 (0.75, 1.61)& 1.000 (1.00, 1.00)& 0.000 (0.00, 0.00)& 1.000 (1.00, 1.00)&1.6s (1.4s, 2.1s)\\
    & MCMC &
 0.371 (0.24, 0.51)& 1.151 (0.77, 1.62)& 1.000 (1.00, 1.00)& 0.000 (0.00, 0.00)& 1.000 (1.00, 1.00)&3m 18s (3m 10s, 4m 8s)\\
    & GSSL &
 0.365 (0.23, 0.51)& 1.144 (0.74, 1.63)& 1.000 (1.00, 1.00)& 0.000 (0.00, 0.00)& 1.000 (1.00, 1.00)&11.5s (10.6s, 14.4s)\\
    [.4em]
    \hline 
    % --- Block
    \rule{0pt}{1\normalbaselineskip}
    \multirow{3}{*}{\textit{Setting 3}}
    & GSVB &
 0.395 (0.28, 0.54)& 1.263 (0.88, 1.79)& 1.000 (1.00, 1.00)& 0.000 (0.00, 0.00)& 1.000 (1.00, 1.00)&2.2s (1.5s, 7.2s)\\
    & MCMC &
 0.398 (0.28, 0.54)& 1.253 (0.86, 1.78)& 1.000 (1.00, 1.00)& 0.000 (0.00, 0.00)& 1.000 (1.00, 1.00)&3m 19s (3m 9s, 4m 28s)\\
    & GSSL &
 0.396 (0.28, 0.54)& 1.269 (0.89, 1.79)& 1.000 (1.00, 1.00)& 0.000 (0.00, 0.00)& 1.000 (1.00, 1.00)&11.8s (10.6s, 13.9s)\\
    [.4em]
    \hline 
\end{tabular} %
}}

    \caption{Companion of Group-sparse Bayesian variable selection methods}
    \label{tab:bvs_comprison}
\end{table}

\red{TODO, comparison of coverage}


\newpage
\section{Extension of Variational Family}

We would like to capture the dependence within the groups. We therefore consider variational family
\begin{equation}
    \Q_D = \left\{ Q_D =  \bigotimes_{k=1}^M \left[ 
    \gamma_k\ N\left(\mu_{G_k}, \Sigma_k \right) + (1-\gamma_k) \delta_0 \right] \right\}
\end{equation}
where $\Sigma_k \in \R^{m_k \times m_k}$ is a positive semi-definite matrix. We proceed as before deriving the update equations for 

\subsection{Updates for $\mu_{G_k}$ and $\Sigma_{k}$}

We follow a similar process as before, recall from \eqref{eq:mu_sigma_1}, we have:
{\allowdisplaybreaks
\begin{align}
& \E_{Q_D  | z_K = 1}  \left[ 
    - \ell(\D; \beta) + \log \frac{dQ_D}{d\Pi}(\beta) 
\right]  \nonumber \\
& = 
    \E_{Q_D | z_K = 1} \left[ 
	\frac{1}{2\tau^2} \tr \left( X^\top X \beta \beta^\top \right) 
	- \frac{1}{\tau^2} \langle y, X_{G_K} \beta_{G_K} \rangle 
	+ \log \frac{dQ_{D,K}}{d\Pi_K}(\beta_{G_K})
    \right] + C \label{eq:QD_mu_sigma}
\end{align}
}
Approaching each term separately using previous results we have,
\begin{align}
    \E_{Q_D | z_K = 1} & \left[ 
	\tr \left( X^\top X \beta \beta^\top \right) 
    \right] \\
=&\
    \sum_{i \in G_K} \left( 
	\sum_{j \in G_K} (X^\top X)_{ji} 
	    \E_{Q | z_K = 1} \left[ \beta_j \beta_i \right] 
	+ 
	2 \sum_{j \in G_K^c} (X^\top X)_{ji} 
	    \E_{Q | z_K = 1} \left[ \beta_j \beta_i \right] 
    \right ) + C \nonumber
\end{align}
with
\begin{equation}
    \E_{Q | z_K = 1} \left[ \beta_j \beta_i \right] = \begin{cases}
	\Sigma_{i, j} + \mu_{i} \mu_{j} & \quad i,j \in G_K \\
	\gamma_J \mu_{j}\mu_{i} 	& \quad i \in G_K, j \in G_J, J \neq K
    \end{cases}
\end{equation}
The middle term is trivial and finally,
\begin{equation}
    \E_{Q_D | z_K = 1 } \left[ \log \frac{dQ_K}{d\Pi_K} (\beta_{G_K}) \right]
    = -\frac{1}{2} \log \det \Sigma_K + \lambda E_{Q_D | z_K = 1} \| \beta_{G_K} \|
\end{equation}
Using \eqref{eq:mu_sigma_upper} to upper bound the previous display gives the objective function for $\mu_{G_K}$ and $\Sigma_K$. Formally,
\begin{equation} \label{eq:QD_mu_sigma_obj}
\begin{aligned}
    \frac{1}{2\tau^2} 
    \mu_{G_K}^\top X_{G_K}^\top X_{G_K} \mu_{G_K}
+
    \frac{1}{2\tau^2} 
    \tr( X_{G_K}^\top X_{G_K} \Sigma_K)
+
    \frac{1}{\tau^2} 
    \sum_{J \neq K} 
	\gamma_J \mu_{G_K}^\top X_{G_K}^\top X_{G_J} \mu_{G_J} \\
-
    \frac{1}{\tau^2}
    \langle y, X_{G_K} \mu_{G_K} \rangle 
-	
    \frac{1}{2} \log \det \Sigma_K 
+
    \lambda \left( \sum_{i \in G_K} 
	\sigma_i^2 + \mu_i^2
    \right)^{1/2} + C
\end{aligned}
\end{equation}
\red{Some trickery.} Following \citep[pg. 119]{Seeger1999} we show that only $2m_k$ free parameters are needed to describe the optimum. Naively we would expect to need $(m_k + 1)m_k /2 $ for the covariance $\Sigma_K$ and $m_k$ for $\mu_k$. 

To show this, let $\Psi = X_{G_K}^\top X_{G_K}$ and $\nu = \lambda (\sum_{i \in G_K} \sigma^2_i + \mu^2_i)^{1/2}$, further write,
\begin{equation}
    \pi = \frac{\partial \nu}{\partial \mu_{G_K}}, \quad W = 
    \frac{\partial \nu}{\partial \Sigma_{K}} = \diag \frac{\partial \nu}{\partial \sigma_{G_K, i}^2}
\end{equation}
Differentiating \eqref{eq:QD_mu_sigma_obj} with respect to $\mu_{G_k}$ gives
\begin{equation}
    \frac{1}{2\tau^2} 
    \Psi \mu_{G_K}
+
    \frac{1}{\tau^2} 
    \sum_{J \neq K} 
	\gamma_J \mu_{G_K}^\top X_{G_K}^\top X_{G_J} \mu_{G_J} \\
-
    \frac{1}{\tau^2}
    X_{G_K}^\top y
+
    \pi
\end{equation}
setting to zero and re-arranging gives
\begin{equation}
    \widehat{\mu}_{G_K} = 
 - 2 \Psi^{-1} \left(  \sum_{J \neq K} 
	\gamma_J \mu_{G_K}^\top X_{G_K}^\top X_{G_J} \mu_{G_J} \\
-
    X_{G_K}^\top y
+
    \tau^2 \pi \right)
\end{equation}
Similarly differentiating \eqref{eq:QD_mu_sigma_obj} wrt. $\Sigma_K$ gives,
\begin{equation}
    \frac{1}{2\tau^2} \Psi
-	
    \frac{1}{2} \Sigma_K^{-1}
+
    W
\end{equation}
setting to zero and re-arranging gives,
\begin{equation}
    \widehat{\Sigma}_K = \left( \tau^{-2} \Psi + W \right)^{-1}
\end{equation}
since $W$ is a diagonal matrix we see that the optimal $\Sigma_K$ depends on only $m_k$ parameters. Notably, although the number of free parameters has not been increased from the independent and to the unconstrained covariance case, optimization of $\Sigma_K$ requires the inversion of an $m_k \times m_k$ matrix, which would be time consuming for large $m_k$ (inversion is $O(m_k^3)$).


\subsection{Update of $\gamma_K$}

The update equation for $\gamma_k$ is given by solving,
\begin{equation} \label{eq:QD_update_gamma} 
\begin{aligned}
    \log &\ \frac{\gamma_K}{1-\gamma_K} = 
    \log \frac{\bar{w}}{1-\bar{w}}
+ 
    \frac{m_K}{2}  
+
    \frac{1}{\tau^2} \langle y, X_{G_K} \mu_{G_K} \rangle  \\
+ &\ 
    \frac{1}{2} \log \det \left( 2 \pi \Sigma_K \right)
+
    \log(C_K )
+
    m_K \log (\lambda)
-
\Bigg\{ 
    \lambda \left( \sum_{j \in G_K} 
	\sigma_j^2 + \mu_j^2
    \right)^{1/2}  \\
+ &\
    \frac{1}{2\tau^2}
    \sum_{i \in G_K} \left( 
    \sum_{j \in G_K} \left[
	(X^\top X)_{ii} (\Sigma_{k, ij} + \mu_j \mu_i)
    \right]
+
    % \frac{1}{\sigma^2}
    % \sum_{i \in G_K} \left( 
    2 \sum_{j \in G_K^c} (X^\top X)_{ji} 
	\gamma_{J} \mu_j \mu_i
    \right )
\Bigg\}
\end{aligned}
\end{equation}





\newpage
\section{Application to real data}

\red{TODO}

\newpage
\section{Extension to classification}


\newpage
\bibliography{refs.bib}

\appendix
\numberwithin{equation}{section}


\section{Definitions}
\textbf{Definition} \textit{Kullback-Leibler divergence}.\\ Let $Q$ and $P$ be probability measures on $\Xc$, such that $ Q $ is absolutely continuous with respect to $P$, then the Kullback-Leibler divergence is defined as,
\begin{equation}\label{eq:kl}
\KL(Q \| P) = \int_{\Xc} \log \left( \frac{dQ}{dP} \right) dQ
\end{equation}
where $dQ/dP$ is the Radon-Nikodym derivative of $Q$ with respect to $P$.


\section{Co-ordinate ascent algorithm}

\subsection{Element-wise update equations}

\textbf{Updates for $\mu_{G_K}$}
\begin{equation} \label{eq:mu_update}
\begin{aligned}
    f(\mu_i; & \mu_{-i}, \sigma, \gamma) :=\
    \frac{1}{2\tau^2} \left(
	(X^\top X)_{ii} \mu_i^2 + 
	\sum_{j \in G_K, j\neq i} (X^\top X)_{ji} \mu_j \mu_i
    \right) \\
+ &\
    \frac{1}{\tau^2} \left(
	\bigg( \sum_{j \in G_K^c} (X^\top X)_{ji} \gamma_{J} \mu_j \mu_i \bigg) -
	\mu_i \langle y, X_{:i} \rangle   
    \right)
+
    \lambda \left( \sum_{j \in G_K} 
	\sigma_j^2 + \mu_j^2
    \right)^{1/2} + C
\end{aligned}
\end{equation}
The above expression in turn is minimized via optimization routines.

Using the fact that $ \left( \sum_{j \in G_K} \sigma_j^2 + \mu_j^2 \right)^{1/2} \leq 1 + \sum_{j \in G_K} \sigma_j^2 + \mu_j^2 $, we can obtain a looser upper bound on \eqref{eq:mu_sigma_main} and the resulting expression we need to minimize,
\begin{equation}
\begin{aligned}
    f_2(\mu_i; & \mu_{-i}, \sigma, \gamma) :=\
    \frac{1}{2\tau^2} \left(
	(X^\top X)_{ii} \mu_i^2 + 
	\sum_{j \in G_K, j\neq i} (X^\top X)_{ji} \mu_j \mu_i
    \right) \\
+ &\
    \frac{1}{\tau^2} \left(
	\sum_{j \in G_K^c} \left[ (X^\top X)_{ji} \gamma_{J} \mu_j \mu_i \right] -
	\mu_i \langle y, X_{:i} \rangle   
    \right)
+
    \lambda \mu_i^2 + C
\end{aligned}
\end{equation}
which in turn in minimized when
\begin{equation} \label{eq:mu_analytic}
    \mu_i =
    - \frac{
	\left(\sum_{j \in G_K^c} (X^\top X)_{ji} \gamma_{J} \mu_j \right) +
	\frac{1}{2} \left(\sum_{j \in G_K, j\neq i} (X^\top X)_{ji} \mu_j \right)-
	\langle y, X_{:i} \rangle 
    }{
	(X^\top X)_{ii} +
	2 \tau^2 \lambda 
    }
\end{equation}

Interestingly, if we assume the columns of $X$ are orthogonal, i.e. $ (X^\top X)_{ij} = 0$ for $i \neq j$, and $\lambda = 0$, then \eqref{eq:mu_analytic} can be written as
\begin{equation}
    \mu_i^{\text{ols}} = (X^\top X)_{ii}^{-1} (X_{:i})^\top y
\end{equation}
which we recognize as the ordinary least squares estimator under an orthogonal design. Similarly, when $\lambda > 0$, the minimizer is given by
\begin{equation}
    \mu_i^{\text{rr}} := \left((X^\top X)_{ii} + 2\tau^2 \lambda \right)^{-1} (X_{:i})^\top y
\end{equation}
which we recognize as the solution under the ridge penalty. It follows that the minimizer \eqref{eq:mu_analytic} is given by a ridge term under the assumption of an orthogonal design and some additional term, formally,
\begin{equation}
    \mu_i = \mu_i^{\text{rr}} - \frac{
	\left(\sum_{j \in G_K^c} (X^\top X)_{ji} \gamma_{J} \mu_j \right) +
	\frac{1}{2} \left(\sum_{j \in G_K, j\neq i} (X^\top X)_{ji} \mu_j \right)
    }{
	(X^\top X)_{ii} +
	2 \tau^2 \lambda 
    }
\end{equation}

\textbf{Updates for $\sigma_{G_K}$}
\begin{equation}
\begin{aligned}
    g(\sigma_i;& \mu, \sigma_{-i}, \gamma) :=\
    \frac{1}{2\tau^2} (X^\top X)_{ii} \sigma_i^2
-
    \log{\sigma_i}
+
    \lambda \left( \sum_{j \in G_K} 
	\sigma_j^2 + \mu_j^2
    \right)^{1/2} + C
\end{aligned}
\end{equation}
As before, under the looser upper bound we have,
\begin{equation}
    g_2 (\sigma_i; \mu, \sigma_{-i}, \gamma) :=\
    \frac{1}{2\tau^2} (X^\top X)_{ii} \sigma_i^2
-
    \log{\sigma_i}
+
    \lambda \sigma_i^2 + C
\end{equation}
which is minimized when,
\begin{equation}
    \sigma_i = \left( \frac{(X^\top X)_{ii}}{\sigma^2} + 2 \lambda \right)^{-1/2}
\end{equation}
which we notice does not depend on the other parameters and in turn can be used to initialize $\sigma_i$.



\section{MCMC sampler}

We construct a Gibbs sampler to sample from the posterior distribution. To begin, we note that the likelihood can be expressed as,
\begin{equation}
    p(\D | \beta, z, \tau^2) = \prod \phi \left(y_i; \sum_{k = 1}^M z_k \langle x_{G_k}, \beta_{G_k} \rangle, \tau^2 \right)
\end{equation}
In turn, we can re-write our prior as,
\begin{equation}
\begin{aligned}
    \beta_{G_k} \overset{\text{ind}}{\sim} &\ \Psi(\beta_{G_k}; \lambda) \\
z_k | \theta_k \overset{\text{ind}}{\sim} &\ \text{Bernoulli}(\theta_k) \\
    \theta_k \overset{\text{iid}}{\sim} &\ \text{Beta}(a_0, b_0)
\end{aligned}
\end{equation}
Finally, recall our prior on $\xi = \tau^2$, is
\begin{equation}
    \xi \sim \Gamma^{-1}(\xi; a, b)
\end{equation}
which has density $ \frac{b^a}{\Gamma(a)} \left( \frac{1}{\xi} \right)^{a + 1} \exp\left( -\frac{\beta}{x} \right)$.

To sample form the posterior we:
\begin{enumerate}
    \itemsep0em
    \item Initialize $\beta^{(i)}, z^{(i)}, \theta^{(i)}, \xi^{(i)}$
    \item For $i = 1, \dots, N$
    \begin{enumerate}
	\item For $k = 1, \dots, M$
	\begin{enumerate}
	    \item Sample $\theta^{(i)}_{k} \overset{\text{iid.}}{\sim} \text{Beta}(a_0, b_0)$
	\end{enumerate}
	\item For $k = 1, \dots, M$
	\begin{enumerate}
	    \item Sample $z^{(i)}_{k} \overset{\text{ind.}}{\sim} \text{Bernoulli}(p_k)$
	    where
	    \begin{equation}
		p_k = \frac
		{p(z_k = 1 | \D, \beta, \theta, \xi)}
		{p(z_k = 1 | \D, \beta, \theta, \xi) + p(z_k = 0 | \D, \beta, \theta, \xi)}
	    \end{equation}
	\end{enumerate}
	\item For $k = 1, \dots, M$
	\begin{enumerate}
	    \item Sample $ \beta_{G_k}^{(i)} \sim p(\beta_{G_k} | \D, z, \beta^{(i-1)}, \xi)$
	\end{enumerate}
    \item Sample $\xi \overset{\text{iid.}}{\sim} \Gamma^{-1}(a + 0.5n, b + 0.5 \| y - X (\beta \circ z) \| ^2 )$
    \end{enumerate}
\end{enumerate}


\end{document}
