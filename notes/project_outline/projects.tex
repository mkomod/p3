\documentclass{beamer}
\usepackage{amsmath}
\usetheme{CambridgeUS}
% \usetheme{Boadilla}

\author{}
\title{Outline for Upcoming Projects}
\date{\today}

\newcommand{\R}{\mathbb{R}}
\newcommand{\vs}[1]{
    \vspace{#1pt}

}

\begin{document}

\maketitle

\begin{frame}{Outline}
    \tableofcontents
\end{frame}

% Project #1 Probabilistic CCA
\section{Probabilistic CCA}
\begin{frame}{PCCA. Background}
Simpler to begin with probabilistic PCA
\begin{equation}
\begin{aligned}
    X | Z \sim &\ N(WZ + \mu, \Sigma) \\
    Z \sim &\ N(0, I_{d})
\end{aligned}
\end{equation}
where $W \in \R^{p \times d}, Z \in \R^{d}, \Sigma \in \R^{p \times p}, \mu \in \R^{p}$ and $I_d$ is $d \times d$ identity.
\vs{10}
In turn probabilistic CCA
\begin{equation}
\begin{aligned}
    X_1 | Z \sim &\ N(W_1 Z + \mu_1, \Sigma_1) \\
    X_2 | Z \sim &\ N(W_2 Z + \mu_2, \Sigma_2) \\
    Z \sim &\ N(0, I_d) 
\end{aligned}
\end{equation}
where $Z \in \R^{d}, W_j \in \R^{p_j \times d}, \mu_j \in \R^{p_j}, \Sigma_j \in \R^{p_j \times p_j}$ for $j=1,2$.
\end{frame}

\begin{frame}{PCCA. Background cont.}
Issues with probabilistic PCA and in turn probabilistic CCA,
\begin{itemize}
    \item Label switching / non-identifiability when running MCMC.
    \item Difficulties with interpretation.
\end{itemize}
\vs{10}
Some positives from the probabilistic framing,
\begin{itemize}
    \item Generative model.
    \item Latent variables $Z$ can be used for visualization.
    \item EM algorithms (at least for PPCA) can be very fast.
\end{itemize}
\end{frame}

\begin{frame}{PCCA. Objectives and next steps}
What do we want out of a method?
\begin{itemize}
    \item Visualization
    \item Variable selection
\end{itemize}
\end{frame}


% Project #2
\section{Group-sparse spike-and-slab regression}
\begin{frame}{Group-sparse Regression (GSR). Background}
Regression framework with,
\begin{equation}
    y = X\beta + \epsilon
\end{equation}
where $\beta = (\beta_1, \dots, \beta_p)^\top \in \R^p$, $X \in \R^{n \times p}$, $y \in \R^{n}$ and $\epsilon$ is a noise term. 
\vs{20}
Define the groups, $G_k = \{ G_{k_1}, \dots, G_{k_p} \}$ for $k=1,\dots,M$ to be disjoint sets of indices, such that $ \bigcup_{k=1}^M G_k = \{1, \dots, p \}$.
\end{frame}

\begin{frame}{Group-sparse Regression. Background cont.}
Patterns of sparsity include:
\vs{4}
\begin{itemize}
    \itemsep8pt
    \item \textbf{Coordinate sparsity}: few coordinate of $\beta$ are non-zero.
    \item \textbf{Group sparsity}: few vectors $\beta_{G_k} = (\beta_{G_{k_1}}, \dots, \beta_{G_{k_p}})$ are non-zero.
    \item \textbf{Sparse-group sparsity}: few vectors $\beta_{G_k} = (\beta_{G_{k_1}}, \dots, \beta_{G_{k_p}})$ are non-zero and the vectors themselves are sparse.
\end{itemize}
\end{frame}

\begin{frame}{GSR. Proposal}
\begin{block}{Project proposal}
    Variational extension to the group-sparse setting.
\end{block}
\vs{10}
Does this fit our objectives (or would we ideally seek an extension to the sparse group-sparse setting?)
\end{frame}

\begin{frame}{GSR. Outline}
\begin{enumerate}
    \itemsep1em
    \item Extension of linear model to the group-sparse setting
    \item Simulations
    \item Theoretical results?
    \item Extend to logistic / survival
\end{enumerate}
\end{frame}


% Project #3
\section{3D breast cancer imaging data}
\begin{frame}{3D Breast cancer data (BCD), Background}
\begin{itemize}
    \itemsep0.8em
    \item Eric's group have access to 3D imaging data and tumor class labels
    \item The project involves using the data to construct a classification model
    \item The raw  data has been split into 1cm x 1cm x 1cm regions and passed through the lab's radiomics pipeline, giving $K_i = \sum \text{regions}$ realizations for each feature.
\end{itemize}
\end{frame}

\begin{frame}{BCD, Background cont.}
\begin{equation*}
\begin{matrix}
    \text{Sample} & & x_1 & x_2 & \dots & x_{p-1} & x_p \\
		  &&&& \uparrow & \\
    1 		  &&&& K_1 & \\
      		  &&&& \downarrow & \\
		  &&&& \vdots \\
		  &&&& \uparrow & \\
    n 		  &&&& K_n & \\
      		  &&&& \downarrow & \\
\end{matrix}
\end{equation*}
% Each sample $x^{(i)}_{j, k}$
\end{frame}

\begin{frame}{BCD, Next steps}
\begin{itemize}
    \itemsep1em
    \item Exploratory data analysis
    \item Literature review for methods dealing with multiple different realizations of the same features for each sample. 
    \item Model proposals
\end{itemize}
\end{frame}

\section{Comments}
\begin{frame}{Comments}
\begin{itemize}
    \itemsep1em
    \item All project's involve high-dimensional data of some sort.
    \item Unclear how to proceed with data integration project.
    \item Can we combine GSR with the Breast cancer dataset?
\end{itemize}
\end{frame}

\end{document}
